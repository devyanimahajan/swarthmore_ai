\documentclass[11pt]{article}
\usepackage{fullpage}
\usepackage{graphicx}

\title{CS63 Spring 2022\\Final Project Checkpoint}
\author{Devyani Mahajan & Cisco Velasco}
\date{}

\begin{document}

\maketitle

\section{Project Goal}

In this project, we aim to train a convolutional neural network to recognise and
translate the ASL alphabet.

\section{AI Methods Used}

Similar to Lab 6, we will be using a convolutional neural network. Here, we will
be utilising Keras and AITK to build and train our network, as well as evaluate
it based on the produced feature maps.

\section{Staged Development Plan}

Sub-goals:
\begin{enumerate}
  \item prepare data: we are using a dataset from Kaggle, and are converting the visual
data to a csv file in order to upload it to our jupyter notebook and analyse the
information in one-hot form
  \item run the neural network succesfully: we want to make sure our data is in optimal
form and we can run our convolutional neural network with no errors to produce
appropriate and helpful feature maps and visualisations of performance
  \item optimise our neural network: we want to select the optimal layer configurations
to achieve the highest possible accuracy for our neural network
\end{enumerate}

Stretch goals:
\begin{itemize}
  \item motion capture: we hope to use our neural network to create a live motion capture
system to translate the ASL alphabet in real time
  \item differentiate between I, J and Z: the ASL signs for J and Z use movement, which
makes them difficult to represent statically. In particular, the sign for J is
statically similar to the sign for I. we hope to find a way to methodologically
differentiate between the two signs with some degree of accuracy
\end{itemize}


\section{Measure of Success}

We will measure success for our neural network through the value accuracy of
our network. We aim to achieve an accuracy of above 80% (accounting for the
difficulties that arise from the letters J and Z). The higher our accuracy, the
greater our success.

\section{Plans for Analyzing Results}

As mentioned above, we can analyse and interpret our results by our val accuracy
measures. In addition to this, we can analyse our feature maps in order to determine
sources of potential error and points of success.

\end{document}
